\documentclass[a4paper,12pt]{article}
\usepackage[utf8]{inputenc}
\usepackage{graphicx}
\usepackage[affil-it]{authblk}
\graphicspath{{images/}}
\usepackage{listings}
\usepackage{indentfirst}
\usepackage{varwidth}
\setlength{\parskip}{10pt}

\begin{document}

\begin{titlepage}
	\begin{center}
		\textbf{\huge Layered Architecture}

		\vspace{0.5cm}

		{\large Software Architecture}

		\vspace{1.5cm}

		\includegraphics[width=0.7\textwidth]{Img/Pradita Logo.png}

		\vspace{1.5cm}

		\textbf{\large Disusun oleh:}

		\vspace{0.5cm}

		\begin{varwidth}{\textwidth}
			\begin{itemize}
				\item Muhammad (2110101031)
				\item Austin Nicholas T. (2110101003)
				\item Darren Valentio (2110101009)
			\end{itemize}
		\end{varwidth}

		\vspace{3.5cm}

		\textbf{\large TEKNIK INFORMATIKA}

		\vspace{0.5cm}

		\textbf{\large PRADITA UNIVERSITY}

		\vspace{0.5cm}

		\textbf{\large 2023}
	\end{center}
\end{titlepage}

\section*{Definisi \textit{Layered Architechture}}

Pola arsitektur layered adalah pola n-tiered di mana komponen disusun dalam lapisan horizontal. Ini adalah metode tradisional untuk merancang sebagian besar perangkat lunak dan dimaksudkan untuk pengembangan mandiri sehingga semua komponen saling berhubungan tetapi tidak saling bergantung.

\includegraphics{Img/Layering Architecture.jpg}

Seperti yang ditunjukkan pada gambar, layering biasanya dilakukan dengan mengemas fungsionalitas khusus aplikasi di lapisan atas, penyebaran fungsionalitas spesifik menjadi lapisan bawah dan fungsionalitas yang membentang di seluruh domain aplikasi di lapisan tengah. Jumlah lapisan dan bagaimana lapisan-lapisan ini disusun ditentukan oleh kompleksitas masalah dan solusinya.

Di sebagian besar arsitektur berlapis, ada beberapa lapisan (atas ke bawah):

\begin{itemize}
	\item \textbf{The application layered:} Berisi layanan spesifik aplikasi.
	\item \textbf{The business layer:} Menangkap komponen yang umum di beberapa aplikasi.
	\item \textbf{The middleware layer:} Lapisan ini mengemas beberapa fungsi seperti pembangun GUI, antarmuka ke basis data, laporan, dan dll.
	\item \textbf{The database/System Software Layer:} Berisi OS, database, dan antarmuka ke komponen perangkat keras tertentu.
\end{itemize}


\section*{Software Architechture Pattern}
Ini adalah pola arsitektur paling umum di sebagian besar aplikasi tingkat perusahaan. Ini juga dikenal sebagai pola n-tier, dengan asumsi n jumlah tingkatan. Contoh Skenario:

\includegraphics{Img/Software Architecture Pattern.png}


\section*{Design Patterns}

Anggap mock-up software design, susunan “stack” nya seperti layered architecture:

\includegraphics{Img/Design Pattern.png}

Setiap layer dari aplikasi terpisah dengan cara penggunaan metode API, namun yang masih saling berhubungan adalah memory handling , karena setiap komunikasi layer akan membawa/mengirim data sehingga akan terjadi alokasi memory dan pada akhirnya membutuhkan memory handling.

Ada 4 bagian dari layered architecture yang di mana setiap layer memiliki hubungan antara komponen yang ada di dalamnya dari atas ke bawah yaitu:

\begin{itemize}
	\item \textbf{The presentation layer:} Semua bagian yang berhubungan dengan layer presentasi.
	\item \textbf{The business layer:} Berhubungan dengan logika bisnis.
	\item \textbf{The persistence layer:} Berguna untuk mengurusi semua fungsi yang berhubungan dengan objek relasional

	\item \textbf{The database layer:} Tempat penyimpanan semua data layer.
\end{itemize}

\includegraphics{Img/Design Pattern 2.png}

\end{document}