\documentclass[a4paper,12pt]{book}
\usepackage[utf8]{inputenc}
\usepackage{graphicx}
\graphicspath{{images/}}
\usepackage{listings}

\begin{document}
	\title{Layered Architechture}
	\author{Muhammad}
	\date{2023}
	\maketitle
	
\section{Latar Belakang}
Penilaian untuk setiap karakteristik berdasarkan kecenderungan alami untuk implementasi tipikal pola layered.
\begin{itemize}
	\item Kemampuan untuk merespon dengan cepat terhadap lingkungan yang terus berubah. (monolitik)
	\item Bergantung pada implementasi pola, penyebaran bisa menjadi masalah. Satu perubahan kecil ke komponen dapat memerlukan redeployment seluruh aplikasi.
	\item Pengembang dapat memberikan pengujian singkat untuk menguji aplikasi sebelum klien menggunakannya
	\item Mudah dikembangkan karena polanya sudah terkenal dan tidak terlalu rumit untuk melakukan implementasinya.
\end{itemize}

\section{Pros Cons}
\subsection{Pros}
\begin{itemize}
\item Mudah untuk diuji karena komponen-komponennya termasuk lapisan khusus sehingga dapat diuji secara terpisah.
\item Sederhana dan mudah diimplementasikan karena secara alami, sebagian besar aplikasi bekerja berlapis-lapis
\end{itemize}

\subsection{Cons}
\begin{itemize}
\item Tidak mudah untuk melakukan perubahan pada lapisan tertentu karena aplikasi merupakan unit tunggal.  
\item Kopling antar lapisan cenderung membuatnya lebih sulit. Hal ini membuatnya sulit untuk diukur. 
\item Harus digunakan sebagai unit tunggal sehingga perubahan ke lapisan tertentu berarti seluruh sistem harus dipekerjakan kembali. 
\item Semakin besar, semakin banyak sumber daya yang dibutuhkan untuk permintaan untuk melewati beberapa lapisan dan dengan demikian akan menyebabkan masalah kinerja.
\end{itemize}


\end{document}

