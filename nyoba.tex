\documentclass[10pt,a4paper]{article}
\usepackage[utf8]{inputenc}
\usepackage[T1]{fontenc}
\usepackage{amsmath}
\usepackage{amssymb}
\usepackage{graphicx}
\begin{document}
\author{Ahmaaad}
\title{Uji Coba Saja Bos}
\date{17 Agustus 1945}

\section{Ahmaaad anak kesayangan tuhan}
\section{Latar Belakang}
Pada mulanya pengembangan perangkat lunak menyatukan fungsi-fungsi dari \textit{graphical user interface} (GUI) dan pengelolaan data ke dalam satu kode tanpa memisahkan mereka sesuai dengan perhatian (\textit{concerns}) mereka masing-masing. 
Konsekuensinya, pola tersebut akan menimbulkan masalah ketika \textit{developer} diminta untuk membangun aplikasi  skala besar,  misalnya aplikasi yang menolong pengguna berinteraksi dengan dataset yang besar dan kompleks. Kode program akan menjadi lebih tidak terstruktur (\textit{spaghetti code}) dan sulit untuk dipahami. 
Sebagai solusi, kode program perlu dibagi ke dalam komponen-komponen sesuai dengan perhatian mereka (\textit{separation of concerns}). 
Arsitektur Model-View-Controller (MVC) kemudian diajukan untuk membagi kode program ke dalam tiga abstraksi utama: \textit{model}, \textit{view}, dan \textit{controller}.

\section{Arsitektur Model-View-Controller}
Arsitektur MVC adalah pola arsitektur untuk pengembangan \textit{Graphical User Interface} (GUI). Arsitektur tersebut membagi logika progam menjadi 3 bagian yang saling terhubung: Model, View, dan Controller. Skema dari MVC dapat dilihat pada Gambar \ref{fig:mvc}..

\textbf{Model} ditujukan untuk berinteraksi dengan data: menyimpan, memperharui, menghapus, dan menarik data dari database. Model juga digunakan untuk menggagregasi data sesuai dengan logika bisnis yang dijalankan. 

\textbf{View} merupakan presentasi yang ditampilkan ke pengguna yang dengannya pengguna dapat berinteraksi. Misalnya, halaman web, GUI desktop, diagram, \textit{text fields}, \textit{buttons}, dsb.

\textbf{Controller} bertugas untuk menerima input dari pegguna melalui \textit{view} dan meneruskan input tersebut ke model untuk disimpan atau diproses lebih lanjut. Controller juga menarik data dari \textit{model} dan memembetuknya demikian rupa sehingga siap untuk dikirimkan ke \textit{view} untuk ditampilkan ke pengguna.

\subsection{Mantap}
\begin{itemize}
	\item Waw
	\item Sheeesh
\end{itemize}

\end{document}